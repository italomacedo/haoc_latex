%% abtex2-modelo-relatorio-tecnico.tex, v-1.9.7 laurocesar
%% Copyright 2012-2018 by abnTeX2 group at http://www.abntex.net.br/ 
%%
%% This work may be distributed and/or modified under the
%% conditions of the LaTeX Project Public License, either version 1.3
%% of this license or (at your option) any later version.
%% The latest version of this license is in
%%   http://www.latex-project.org/lppl.txt
%% and version 1.3 or later is part of all distributions of LaTeX
%% version 2005/12/01 or later.
%%
%% This work has the LPPL maintenance status `maintained'.
%% 
%% The Current Maintainer of this work is the abnTeX2 team, led
%% by Lauro César Araujo. Further information are available on 
%% http://www.abntex.net.br/
%%
%% This work consists of the files abntex2-modelo-relatorio-tecnico.tex,
%% abntex2-modelo-include-comandos and abntex2-modelo-references.bib
%%

% ------------------------------------------------------------------------
% ------------------------------------------------------------------------
% abnTeX2: Modelo de Relatório Técnico/Acadêmico em conformidade com 
% ABNT NBR 10719:2015 Informação e documentação - Relatório técnico e/ou
% científico - Apresentação
% ------------------------------------------------------------------------ 
% ------------------------------------------------------------------------

\documentclass[
	% -- opções da classe memoir --
	12pt,				% tamanho da fonte
	openright,			% capítulos começam em pág ímpar (insere página vazia caso preciso)
	twoside,			% para impressão em recto e verso. Oposto a oneside
	a4paper,			% tamanho do papel. 
	% -- opções da classe abntex2 --
	%chapter=TITLE,		% títulos de capítulos convertidos em letras maiúsculas
	%section=TITLE,		% títulos de seções convertidos em letras maiúsculas
	%subsection=TITLE,	% títulos de subseções convertidos em letras maiúsculas
	%subsubsection=TITLE,% títulos de subsubseções convertidos em letras maiúsculas
	% -- opções do pacote babel --
	english,			% idioma adicional para hifenização
	french,				% idioma adicional para hifenização
	spanish,			% idioma adicional para hifenização
	brazil,				% o último idioma é o principal do documento
	]{abntex2}


% ---
% PACOTES
% ---

% ---
% Pacotes fundamentais 
% ---
\usepackage{lmodern}			% Usa a fonte Latin Modern
\usepackage[T1]{fontenc}		% Selecao de codigos de fonte.
\usepackage[utf8]{inputenc}		% Codificacao do documento (conversão automática dos acentos)
\usepackage{indentfirst}		% Indenta o primeiro parágrafo de cada seção.
\usepackage{color}				% Controle das cores
\usepackage{graphicx}			% Inclusão de gráficos
\usepackage{microtype} 			% para melhorias de justificação
\usepackage{fancyhdr}
\usepackage{tikz}
\usetikzlibrary{calc}
\usepackage{tikzpagenodes}
% ---

% ---
% Pacotes adicionais, usados no anexo do modelo de folha de identificação
% ---
\usepackage{multicol}
\usepackage{multirow}
% ---
	
% ---
% Pacotes adicionais, usados apenas no âmbito do Modelo Canônico do abnteX2
% ---
\usepackage{lipsum}				% para geração de dummy text
% ---

% ---
% Pacotes de citações
% ---
\usepackage[brazilian,hyperpageref]{backref}	 % Paginas com as citações na bibl
\usepackage[alf]{abntex2cite}	% Citações padrão ABNT
\usepackage[style=abnt]{biblatex}
\addbibresource{export-data.bib}   

% --- 
% CONFIGURAÇÕES DE PACOTES
% --- 
\usepackage{appendix}

% ---
% Configurações do pacote backref
% Usado sem a opção hyperpageref de backref
\renewcommand{\backrefpagesname}{Citado na(s) página(s):~}
% Texto padrão antes do número das páginas
\renewcommand{\backref}{}
% Define os textos da citação
\renewcommand*{\backrefalt}[4]{
	\ifcase #1 %
		Nenhuma citação no texto.%
	\or
		Citado na página #2.%
	\else
		Citado #1 vezes nas páginas #2.%
	\fi}%
% ---

% ---
% Informações de dados para CAPA e FOLHA DE ROSTO
% ---
\titulo{Documento de especificação de requisitos}
\autor{Italo Macêdo do Amaral Costa}
\local{Brasil}
\data{2023, v-1.0.0}
\instituicao{%
  Hospital Alemão Oswaldo Cruz
  \par
  Programa de Apoio ao Desenvolvimento Institucional do Sistema Único de Saúde}
\tipotrabalho{Relatório técnico}
% O preambulo deve conter o tipo do trabalho, o objetivo, 
% o nome da instituição e a área de concentração 
\preambulo{Projeto de Qualificação de Dados Assistenciais (PQDAS)}
% ---

% ---
% Configurações de aparência do PDF final
\fancyhead[L]{
\begin{tikzpicture}[remember picture,overlay]
\draw  let \p1=($(current page.north)-(current page header area.south)$),
      \n1={veclen(\x1,\y1)} in
node [inner sep=0,outer sep=0,below right] 
      at (current page.north west){\includegraphics[scale=1]{header.pdf}};      
\end{tikzpicture}}

% alterando o aspecto da cor azul
\definecolor{blue}{RGB}{41,5,195}

% informações do PDF
\makeatletter
\hypersetup{
     	%pagebackref=true,
		pdftitle={\@title}, 
		pdfauthor={\@author},
    	pdfsubject={\imprimirpreambulo},
	    pdfcreator={LaTeX with abnTeX2},
		pdfkeywords={abnt}{latex}{abntex}{abntex2}{relatório técnico}, 
		colorlinks=true,       		% false: boxed links; true: colored links
    	linkcolor=blue,          	% color of internal links
    	citecolor=blue,        		% color of links to bibliography
    	filecolor=magenta,      		% color of file links
		urlcolor=blue,
		bookmarksdepth=4
}
\makeatother
% --- 

% --- 
% Espaçamentos entre linhas e parágrafos 
% --- 

% O tamanho do parágrafo é dado por:
\setlength{\parindent}{1.3cm}

% Controle do espaçamento entre um parágrafo e outro:
\setlength{\parskip}{0.2cm}  % tente também \onelineskip

% ---
% compila o indice
% ---
\makeindex
% ---

% ----
% Início do documento
% ----
\begin{document}

% Seleciona o idioma do documento (conforme pacotes do babel)
%\selectlanguage{english}
\selectlanguage{brazil}

% Retira espaço extra obsoleto entre as frases.
\frenchspacing 

% ----------------------------------------------------------
% ELEMENTOS PRÉ-TEXTUAIS
% ----------------------------------------------------------
% \pretextual

% ---
% Capa HAOC
% ---
\begin{tikzpicture}[overlay,remember picture]
 \node[anchor=north west,inner sep=0pt]at ([xshift=-0.5cm,yshift=1cm]current page.north west) {\includegraphics[scale=1]{header.pdf}};
 \end{tikzpicture}

\begin{historicorevisao}
\begin{table}[!h]
\centering
\begin{tabular}{|p{4cm}p{4cm}p{4cm}p{4cm}|}
\hline
\multicolumn{1}{|p{4cm}|}{\textbf{Projeto}}       & \multicolumn{3}{p{12cm}|}{Projeto de Qualificação de Dados Assistenciais (PQDAS)} \\ \hline
\multicolumn{1}{|p{4cm}|}{\textbf{Gestor do projeto}}       & \multicolumn{3}{p{12cm}|}{Tathiana Soares Machado Velasco} \\ \hline
\multicolumn{1}{|p{4cm}|}{\textbf{e-mail}}       & \multicolumn{3}{p{12cm}|}{tvelasco@haoc.com.br} \\ \hline
\multicolumn{1}{|p{4cm}|}{\textbf{Telefone}}       & \multicolumn{3}{p{12cm}|}{(011) 967564441} \\ \hline
\multicolumn{1}{|p{4cm}|}{\textbf{Responsável}}       & \multicolumn{3}{p{12cm}|}{Haliton de Oliveira Junior} \\ \hline
\multicolumn{1}{|p{4cm}|}{\textbf{e-mail}}       & \multicolumn{3}{p{12cm}|}{haoliveira@haoc.com.br} \\ \hline
\multicolumn{1}{|p{4cm}|}{\textbf{Telefone}}       & \multicolumn{3}{p{12cm}|}{(11) 3549-0581} \\ \hline
\end{tabular}
\end{table}
\end{historicorevisao}

\begin{center}
    NOME DO PROJETO
\end{center}

\begin{historicorevisao}
\begin{table}[!h]
\centering
\begin{tabular}{|p{4cm}p{4cm}p{4cm}p{4cm}|}
\hline
\multicolumn{4}{|c|}{\textbf{Histórico de revisões}}                                                                                    \\ \hline
\multicolumn{1}{|c|}{Data}       & \multicolumn{1}{c|}{Autor}        & \multicolumn{1}{c|}{Descrição}                          & Versão \\ \hline
\multicolumn{1}{|c|}{16/11/2022} & \multicolumn{1}{c|}{Italo Macêdo} & \multicolumn{1}{c|}{Versão inicial do documento}        & 0.1    \\ \hline
\multicolumn{1}{|c|}{22/02/2023} & \multicolumn{1}{c|}{Italo Macêdo} & \multicolumn{1}{c|}{Versão pós-diligência consultoria.} & 0.9    \\ \hline
\multicolumn{1}{|c|}{24/04/2023} & \multicolumn{1}{c|}{Italo Macêdo} & \multicolumn{1}{c|}{Versão pós-diligência 2 consultoria.} & 1.0    \\ \hline
\end{tabular}
\end{table}
\end{historicorevisao}
\cleardoublepage

% ---
% inserir lista de ilustrações
% ---
\pdfbookmark[0]{\listfigurename}{lof}
\listoffigures*
\cleardoublepage
% ---

% ---
% inserir lista de tabelas
% ---
\pdfbookmark[0]{\listtablename}{lot}
\listoftables*
\cleardoublepage
% ---

% ---
% inserir lista de abreviaturas e siglas
% ---
\begin{siglas}
\item[ANS] Agência Nacional de Saúde Suplementar
\item[API] Application Programming Interface
\item[CID-10] Classificação Estatística Internacional de Doenças e Problemas Relacionados a Saúde-10
\item[CBO] Classificação Brasileira de Ocupações
\item[COPISS] Comitê de Padronização da Saúde Suplementar 
\item[DIPRO] Diretoria de Produtos da ANS
\item[FHIR] Fast Healthcare Interoperability Resources
\item[GEMOA] Gerência de Monitoramento Assistencial - DIPRO/ANS
\item[GEPIN Gerência de Padronização, Interoperabilidade e Análise de Informação - DIDES/ANS
\item[GETI] Gerência de Tecnologia de Informação - DIDES/ANS
\item[HAOC] Hospital Alemão Oswaldo Cruz
\item[HL7] Health Level 7
\item[IDSS] Índice de Desempenho da Saúde Suplementar
\item[ISO] International Organization for Standardization	
\item[JSON] Notação de Objetos JavaScript (do inglês JavaScript Object Notation)
\item[MS] Ministério da Saúde
\item[PROADI] Programa de Apoio ao Desenvolvimento Institucional do SUS
\item[RNDS] Rede Nacional de Dados em Saúde
\item[PQDAS] Padronização e Qualificação dos Dados Assistenciais da Saúde Suplementar
\item[SIP] Sistema de Informação de Produtos da ANS
\item[SUS] Sistema Único de Saúde
\item[TISS] Troca de Informações em Saúde Suplementar
\item[TUSS] Terminologia Unificada da Saúde Suplementar  
\item[XML] Extensible Markup Language
\end{siglas}
% ---

% ---
% inserir o sumario
% ---
\pdfbookmark[0]{\contentsname}{toc}
\tableofcontents*
\cleardoublepage
% ---


% ----------------------------------------------------------
% ELEMENTOS TEXTUAIS
% ----------------------------------------------------------
\textual

% ----------------------------------------------------------
% Introdução (exemplo de capítulo sem numeração, mas presente no Sumário)
% ----------------------------------------------------------
\chapter{Objetivo deste documento}
Este documento tem como finalidade descrever os requisitos a serem atendidos conforme documento de visão: funcionais e não funcionais. Tais informações serão descritas nas seções a seguir.

\end{document}